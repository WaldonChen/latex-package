% !TEX root =  main.tex

\paragraph{Notation}
Insert manuscript body here. Use \verb|\gls{\cdot}| for
repeated glossary, acronym, and notation items, 
e.g.,~use \verb|\gls{abm}| for the first time,
you get ``\gls{abm}.'' Use it again, you get ``\gls{abm}.''
Similarly, you can use it on glossary items 
(e.g.,~use \verb|\glspl{al}| for plural ``\glspl{al}'')
and mathematical notations 
(e.g.,~use \verb|\gls{state}| or \verb|$\gls{state}$|
for \gls{state}).

\paragraph{Multi-line equation}
Proper way to handle multi-line equations:
\begin{verbatim}
\begin{equation}
  \begin{aligned}
    x(t+1)&=Ax(t)+Bu(t)+\omega(t),\\
    y(t)&=Cx(t)+Du(t)+\nu(t).
  \end{aligned}
\end{equation}
\end{verbatim}
See output:
\begin{equation}
  \begin{aligned}
    x(t+1)&=Ax(t)+Bu(t)+\omega(t),\\
    y(t)&=Cx(t)+Du(t)+\nu(t).
  \end{aligned}
\end{equation}

\paragraph{Citation} Use \verb|\cite{}| or \verb|\citep{}|
depending on the environment.

\paragraph{Root} Insert the following at the beginning
of every sub-file (replace \verb|<main file name>|)
with the main file name you use:
\begin{verbatim}
	% !TEX root =  <main file name>.tex
\end{verbatim}













